\documentclass{article}
\usepackage{graphicx}
\usepackage{amstext}
\usepackage{amsmath}
\usepackage{enumerate}
\usepackage{fancyhdr}

\newcommand{\thetitle}{Shady: A Dynamic Instrumentation Tool for Failure Oblivious Computing}
\newcommand{\theauthor}{Andrew Haven (ahaven), Carl Case (cbcase)}

\title{\thetitle}
\author{\theauthor}

\pagestyle{fancy}
\fancyhead[L]{Haven, Case}
\fancyhead[C]{Shady}
\fancyhead[R]{\thepage}


\newcommand{\code}[1]{\texttt{#1}}

\begin{document}

\maketitle

\section{Introduction} % (fold)
\label{sec:Introduction}

Failure oblivious computing \cite{Rinard04enhancingserver} observes that most programs are built to handle errors.  The use of exceptions and error checking routines means that many errors do not propagate more than a short distance through the code, particularly in the case of request-response programs where errors tend not to escape a single iteration of the loop.  The failure oblivious approach is to take errors that would traditionally be security vulnerabilities or terminate the program and discard the offending instruction, manufacturing values if necessary.  Many programs are able to work under this paradigm and critical security vulnerabilities can be avoided.

We present Shady, a dynamic instrumentation tool built on the DynamoRIO instrumentation platform \cite{Bruening04efficient}. The tool provides failure obliviousness at runtime by modifying program execution to avoid a large class of memory errors. We show that Shady can prevent a dangerous double free exploit and also avoid the heap overrun in the Pine mail user agent that the Failure Oblivious compiler used to demonstrate its effectiveness.

\section{Background} % (fold)
\label{sec:Background}

%   What is DynamoRIO?
%       Why Was DynamoRIO useful and a good fit for our project?  
%       What were the alternatives?
%       What were our impressions and experiences when using it?
%   What were our goals?

In the original paper describing failure oblivious computing, the authors build their system on top of a safe C compiler that can track each reference and ensure that it lands in bounds. This approach has two problems:
\begin{enumerate}
\item You can only detect errors in code the compiler sees. In the best case, you need to get the source for all system libraries. In the worst case, your program calls into closed-source libraries that simply cannot be checked.
\item You have to hack the compiler, which can be tricky work.
\end{enumerate}
Our goal is to provide similar benefits as the failure oblivious compiler but do so entirely in a runtime tool. By inserting memory checks at runtime, we obviate the need for source code and do not need to touch the compiler.

Of course, much of the information available to the compiler is gone at runtime. In particular, we cannot do any reference tracking at runtime. All the tool sees are loads and stores. Thus, one of the goals of Shady is to investigate providing failure obliviousness without any reference information.

We use DynamoRIO to implement our tool.  DynamoRIO is a fast and extensible system for dynamic instrumentation, similar to PIN \cite{Luk05pin:building} and Valgrind \cite{Nethercote07valgrind:a}. There is a Memcheck \cite{Nethercote05memcheck} clone called Dr. Memory built on DynamoRIO with better performance (generally 2x faster). The DynamoRIO API is reasonably well-documented and has good support for x86 instruction modification.

%We looked to prevent the following types of errors:
%\begin{enumerate}
%\item Heap overrun
%\item Stack overrun
%\item Double Free
%\item Bad pointer dereference (e.g., a null pointer)
%\end{enumerate}

%\begin{enumerate}
%    \item[Heap Overflow] Heap overflows occur when a read or write goes beyond the end of a dynamically allocated buffer.  To avoid this we pad the end of every heap block with a red zone, containing sentinel values to detect when such an operation occurs.  We expect most of these to be short, as in the case of a string copy which will terminate at the first null character.
%   \item[Stack Overflow] A program may read or write outside of a stack-allocated buffer, which can overwrite control-flow information such as the base pointer and return address.  This is a common case for many security vulnerabilities.
%    \item[Double Free] A program might try to free the same memory twice.  This is trivial to prevent.
%\end{enumerate}

% section Introduction (end)

\section{Design} % (fold)
\label{sec:Design}
%   What did we implement?
%       What and how did we implement our FO instrumentation?
%           Red zones.
%           Trace all malloc/free
%           Trace all loads/stores
%       What did we choose to avoid implementing?
%           Stack
%       What did we find easy to implement?  hard?

At a high level, Shady performs the following:
\begin{itemize}
\item For each heap allocation, add a red zone before and after with a word-sized \textit{sentinel value}
\item On each memory reference, check that the (word-aligned) address is addressable and does not contain the sentinel value
\item If the previous check fails, thrown out stores and synthesize values for loads
\end{itemize}

The implementation involves instrumenting all loads, stores, and calls to \code{malloc} and \code{free}.  DynamoRIO provides instrumentation hooks to assist in this, which is straightforward overall.  DynamoRIO also provides the ability to save the state of the program, call into instrumentation code, and then return to the original program.  This was invaluable for prototyping, but is not fast enough for production use as we describe later.

Instrumenting \code{malloc} and \code{free} involves wrapping function calls with calls to our instrumentation code.  DynamoRIO provides a facility for function wrapping which automatically calls instrumentation routines before and after the call.  We wrapped all calls to \code{malloc}, \code{free}, \code{calloc} and \code{realloc}.  These calls modify the arguments to allocate space for our pre- and post-red zones.  We used values of 2 words and 4 words respectively for their sizes.  These red zones are filled with our word-sized sentinel value to detect invalid reads and writes. We use a hashmap to track all allocated regions so we can rewrite the parameter to free and also throw away double free.

Instrumenting loads and stores involves taking advantage of DynamoRIO's interpreter/JIT.  An event is triggered whenever a basic block is loaded, which will call any registered callbacks.  These callbacks can then modify the instructions before they are run.  The callback may be called many times for the same block because it will be called each time the block is run until it is added to the trace cache and also for fault address translation.  There is an option to always save basic blocks to the cache, which speeds up the program but incurs a memory overhead.  We observed that it is now 2012 and opted for the additional memory overhead.

The instrumentation for loads and stores makes a call before the memory reference if the instruction reads or writes memory.  That call word-aligns the address and then checks if the memory is readable and whether it is a sentinel value.  In either of these cases, if the operation is a write the operation is skipped and the program moves to the next instruction.  If the operation is a read, then the target of the read is filled with a manufactured value.  We opted for the increasing sequence $[0,1,2,3,\dots]$.  We stored a separate counter for each faulting instruction.

We hoped to find a way to implemented isolation of stack frames to prevent buffer overruns on the stack.  However, we were unable to implement it at this time.  One solution we examined was saving the return address and base pointers to a second protected stack and restoring them on return.  We also looked at inserting red zones into the stack.  In both cases we found it near impossible to detect function calls when operating at the instruction level.  We found that even in modest programs the sequence of calls and returns did not form a properly nested tree.  We leave this to future work.

% section Design (end)

\section{Results} % (fold)
\label{sec:Results}
%   How well did it work?
%       Prevent trivial issues like double-free.
%       What programs did we test?
%           Pine
%       What were our results?
%       How well did it work?
%           Pine reads below recently freed memory?
%           rep stors

\subsection{Failure Obliviousness}

We began by running our tool on a small set of trivial programs, each with a single, obvious error. These programs allowed us to validate that Shady can prevent small heap buffer overruns, double free and null dereference.

Next, we installed a setuid binary with a double free bug. As part of previous classwork, we wrote an exploit program that could use the second free to modify control flow and open a root shell. Since the vulnerable program used a private malloc implementation, we added support to Shady to instrument client functions named something other than malloc / free (as long as they obey the same interface). We then ran the exploit program under Shady. The second call to free is skipped, and the vulnerable program exits cleanly.

Finally, we tested Shady on Pine version 4.44, one of the vulnerable programs used in the original failure oblivious paper. It has a bug in the code that calculates the length of a buffer allocated for the sender's address on email messages when displayed in the message list.  It underestimates the size when a large number of escaped quotes are included inside the quoted string.  This causes the program to crash when displaying the mailbox and it will continue to crash until the mail file is edited to remove the offending message.

When run under Shady, Pine is able to correctly display the message, as the writes beyond the end of the buffer are discarded and when read the sentinel is seen to be zero, which properly terminates the (now truncated) string.  The program is then able to continue operating normally.

% We observed a number of odd behaviors and weird bugs when first testing Shady on ``real'' programs rather than toys. A simple bug that we had to address was the use of the \code{rep stors} instruction, which writes the contents of \code{eax} into \code{edi} a number of times equal to the contents of \code{ecx} and incrementing or decrementing \code{edi} by four bytes each time depending on the direction flag.  There are options in DynamoRIO to express this operation as a loop, but it would involve converting our program to the DynamoRIO Multi-Instrumentation Manager framework.  We judged it would be infeasible to make these changes within our time frame. Instead we ignore all \code{rep} instructions.

% A more complicated bug that we found ourselves running into was that when loading Pine we observed behavior that the program appeared to request a buffer from \code{malloc} and then would write and read memory eight bytes before the start of the buffer.  When we had red zones enabled the write and read would hit the red zone and the result would be zero, causing an eventual error in the program.  If we disabled red zones the program would still read before the beginning of the buffer but would complete successfully.  We solved this by eliminating our pre-buffer red zone during our tests.  Further work would be to investigate this issue further.

\subsection{Performance}

One of the goals of the sentinel value scheme for error detection is it can be very efficient in the common case of no error. Unfortunately, we did not have time to explore DynamoRIO's support for fast, inline code modification. Instead, for ease of debugging, we used its ``clean call'' instrumentation, which saves all (non-floating point) register state and then does a procedure call to the instrumentation function. This is of course an expensive operation, and Shady performs a clean call for every memory reference. Thus, we expect very high overhead when using Shady.

We were not disappointed. As a simple benchmark, we use \code{tar} to extract a gzipped archive with 11 files and is approximately 3MB. We ran \code{tar} on its own; under DynamoRIO with no instrumentation; and under Shady. The results are in Table \ref{tab:perf}. As you can see, the slowdown is enormous. We were a little disappointed to see that there was a 5x slowdown under DynamoRIO itself, as its authors claim much better performance.

\begin{table}
\caption{Shady Slowdown}
\begin{center}
\begin{tabular}{ l | l l l }
  & Time & Std. Dev. & Slowdown \\
\hline
    Baseline       &   0.0368s      &    0.0012s   & 1x \\
    DynamoRIO       &    0.1833s      &    0.004s  & 5x   \\
    Shady       &      27.62s      &    0.0102s  & 750x \\

\end{tabular}
\label{tab:perf}
\end{center}
\end{table}


Remarkably, Pine is still quite useable under Shady. There is little noticeable latency beyond a long start up time.

% section Results (end)

\section{Future Work} % (fold)
\label{sec:Future Work}
%   Future work
%       What do we think could be easily implemented in the future?
%       What would be hard to do?

There are a few major deficiencies with our system.  One is that the system is slow, using the easiest instrumentations, but not the fastest.  A second is that we do not protect stack operations in any way.  A third is that we do not do sound checking of memory.  A final issue is that our code did not work properly with all the code we tested and needs to be extended to handle more cases.

In terms of instrumentation size we believe the greatest gain will come from splitting the memory checks into a fast-path and slow-path check.  The fast-path can be inlined and only call the slow path if a sentinel is encountered.  This should speed up the program greatly because it will not save application state on every memory reference.  DynamoRIO is a fast back end and so the instrumentation we build on top should be able to achieve reasonable speed.

Stack memory should be protected in a similar way to heap memory.  This will protect against a much larger class of errors.  We were unable to implement this in our time frame, and there are hurdles associated with doing this at runtime as discussed previously.

In our implementation we cut corners by choosing a four-byte sentinel value and assuming the client program would not use that value in normal operations.  The correct implementation would be to check whether the memory in question belongs to a red zone whenever a sentinel is encountered by tracking all positions where a red zone should exist.  We were unable to implement this in our time frame.

Finally, we need to extend the code to handle more cases.  For example, our handling of \code{realloc} may not exactly match the semantics of the original call, which can be quite subtle.  We also have some instructions we chose not to support such as \code{rep}.  DynamoRIO has support for all these, as evidenced by Dr. Memory, however the implementation of support for the full x86 instruction set is non-trivial.

% section Future Work (end)

\section{Conclusion} % (fold)
\label{sec:Conclusion}
%   Conclusions
%       What did we learn:
%           x86 is really tricky.

This project was an valuable exploration of dynamic instrumentation tools.  We are both more comfortable with dynamic analysis and aghast at the x86 instruction set and the difficulties of instrumenting it.

The project proved to be a reasonable size to tackle in the abstract, but the implementation details were numerous.  We wonder if future students could extend this project to tackle the feature we were unable to address.

Our code is online at \texttt{https://github.com/periodic/Shady}.

% section Conclusion (end)

\bibliographystyle{plain}
\bibliography{references}

\end{document}
